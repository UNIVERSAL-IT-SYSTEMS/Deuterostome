\documentclass[12pt]{article}

\usepackage[dvips]{geometry}

\usepackage{pdf}
\usepackage{hyperref}

\usepackage{natbib}
\bibliographystyle{plainnat}

\usepackage{redefs}
\renewcommand{\bibsection}{%
  \section*{\refname}%
  \addcontentsline{toc}{section}{\refname}%
}

\title{
  \vspace*{-2\baselineskip}
  D-machine 3.0 Extensions and DVT interface
}
\author{Alexander Peyser}

\begin{document}

\pagenumbering{roman}
\maketitle\vspace{-4\baselineskip}
\vspace{\stretch{1}}\tableofcontents\vspace{\stretch{1}}
\newpage

\setlength{\parindent}{0pt}
\setlength{\parskip}{6pt}
\pagenumbering{arabic}

\section{Introduction}
\label{sec:intro}

The 3.0 version of the d-machine introduces a fundamental shift from a
single Mac OS system to a clustered Unix approach.  This has demanded
a significant rewrite of the underlying operating system interface and
an extension into the world of tcp connectivity. The `dvt,' which
previously was both an interface and a processor, has been replaced
with a dedicated interface and `dnode' manager.  The dnodes handle all
heavy processing independently, and can be viewed as threads within
the larger d-machine.

From an interface perspective, the `eye' and `hand' have been
translated from a Mac OS-centric system handled directly by dvt d-code
to an X-windows based system.  The eye is still handled directly by
the d-code, but has been extended into a group of eyes, one for each
dvt and dnode, functioning independently.  The hand is now composed of
an emacs frame combining emacs lisp code and d-code in the dvt and the
dnodes, for which it acts as an agent.

Additionally, interfaces that will be discussed are the DVT macro bar,
`TheHorses' and the Tiny Window Manager (twm) elements of the
interface. In term of a high-level model, the d-machine has been
distributed over several machines and tools external to the d-machine
proper, to handle the additional complexity due to clustering and to
re-use open source code instead of recoding several internal elements.

A separate manual, ``Emacs \& D-machine: d source files and dvt''
\citep{dvt-emacs}, stored as \verb$dvt-emacs.tex$ and
\verb$dvt-emacs.pdf$, covers the details specifically of the emacs
interface to the dvt.  This manual will focus on overall user function
and relevant d-code, integrating the usage of the system.

\textbf{NB:} On G5 Panther machines, all references in this manual to
the Alt key must be replaced with the Apple key.

\section{System Overview}
\label{sec:overview}

All primary source code resides in \texttt{/mnt/Lab1/dm-3.0/}, which
at the local UM cluster is a shared NSF mount.  For other
installations, it can be on a local machine which is rsync'd for
updates. To build an architecture specific d-machine, \verb$make$ is
called with one of the following paramters: \verb$osx$ for osx G5's,
\verb$g4$ for linux G4's, or \verb$g5$ for linux G5's (which is not
yet implemented).  This builds a directory \texttt{/mnt/Lab1/dm-3.0/arch/}
and a link to it in \texttt{/mnt/Lab1/dm-3.0-arch/}, where arch is
osx, g4 or g5.

Inside this directory, all d-scripts, bash-scripts, perl-scripts and
lisp code is inserted, and the executables \texttt{dvt} and
\texttt{dnode} are placed.  This directory
(\texttt{/mnt/Lab1/dm-3.0-arch/}) should be your current directory
when you run any d-machine associated code.

\subsection{Environment}
\label{sec:env}

The d-machine can be built for linux running on a powerpc (g4), on a
g5 when the 64-bit compiler is production ready, and under osx on a
g5. The code is almost identical under all these architectures; osx,
being a version of BSD unix, has a few parameter changes, in addition
to some Apple `tweaks,' to be kind, in precompiliation which had to be
worked around. Emacsen under which the dvt works include versions
21.3.1 and 21.3.50.1. Each version uses optimized compilation flags,
benchmarked primarily for double-precision floating point operations.

Under both linux and osx, dvt, dnodes and emacs use X-windows as the
graphics server, managed by twm.  Under linux, X is started by
\verb$startx$, and on osx the DarwinX button on the doc is used to
start it. For the latter case there is the option of rootless or
not. When not rootless, the X windows run directly over aqua; in the
other case, an X root window replaces the aqua windows, and can be
switched out with \verb$Ctrl-Apple-A$.  Either will work, to preference.
However, the X root window does not interface well with the aqua
screen saver.

Tiny Window Manager puts the buttons and borders around the windows.
This should be part of the initial installation and is customized for
d.  The d windows have a class of ``d\_machine,'' which twm's
\texttt{system.twmrc} recognizes.  An alternate icon manager is used
to collect those windows, and is initially located in the upper
right-hand side of the screen.  This `docks' the windows for
minimization and maximization, and groups d-machine window separately
from other X windows.

\subsection{DVT}
\label{sec:dvt}

The dvt proper exists as a child process of emacs, running under X.
After initializing X and emacs, the dvt is started via the command
\verb$Alt-x dvt$.  See the dvt-emacs manual \citep{dvt-emacs} for a
general description of the interface. The dvt begins by loading
\texttt{dvt\_startup.d}, which initalizes the \verb$userdict$, and
begins the main interface loop, which is partially implemented in d.
During installation, the lisp code is linked to begin the dvt in the
proper directory.

Minimal d-code should be run directly in the dvt, for several reasons:
\begin{enumerate}
  \item The memory allocated, including stacks sizes, is hard-coded in
  c.  To run large jobs, the code must be recompiled. Dnodes have
  flexible memory allocation.
  \item Debugging capabilities are severely limited. In order to
  minimize loss of interface under an error, whenever an \verb$abort$
  event occurs, the stacks and dictionaries are cleared, returning to
  the main interface loop. This leaves little evidence of the cause of
  the error (there is a work-around though, the \verb$debug_abort$
  mechanism in \texttt{dvt\_startup.d}).
  \item Running code in the dvt will interfere with management of
  d-nodes. Simply, the dvt may not be available for handling dnode
  communication while it is processing, and its behaviour may be quite
  complicated with numerous execution stack subsets running. This
  includes TheHorses functionality, and the macro bar.
  \item The dnodes can be combined to multithread operations; the dvt
  shoudle exist, by definition, as a single master thread.
\end{enumerate}

When the dvt starts, the following windows should pop up:
\begin{description}
  \item[TheEye] is the X version of the old eye.  It allows file
  operations, and dictionary, list and array investigations. It is a
  purely mouse driven interface. On the top bar, the node to which it
  belongs is labeled (at first you only have the dvt eye).
  \item[TheHorses] contain all currently connected nodes of the
  d-machine. Initially, it only contains the dvt itself (labeled
  \verb$dvt$). This interface controls the current keyboard owner and
  reflects the busy state of each node. This is also mouse driven.
  \item[DVT macros] allows the integration of mouse-driven and
  keyboard-driven events. One can choose a macro, which appears in the
  emacs dvt window as a d-command. This d-command may look at
  selections in the eye to open files, etc\dots 
  \item[d\_machine Icon Manager] organizes the icons for the
  preceeding windows. This does not include the main emacs window
  which does not, by default, belong to the proper class.
\end{description}


\subsection{Dnodes}
\label{sec:dnode}

Processing occurs in dnodes, each of which can act as a separate
thread of operation, cooperatively or not.  The can exist on the dvt's
host machine, or on any other available machine. Currently, this has
only been implemented for ppc's, but no obvious endian issues exist
for intercommunication.  The dnode host just needs to have the X
libraries required to communicate with the host, but it does not need
to be running an Xserver, itselve.

Each dnode runs as a server on a tcp port (5000+n). A script exists in
the architecture directory: \verb$dnode.sh$. It has one parameter (n),
and start the dnode immediately. If the dnode fails because the port
is currently unavailable, the script will continue trying to start the
dnode at some small interval.  This may occur after a dnode dies and
before the os has finished the cleanup of the port connection, making
the available once again.

The dnode itself will respond to a \verb$HUP$, \verb$QUIT$ or
\verb$TERM$ signal (such as by: \verb$killall -HUP dnode$) by shutting
down cleanly, and responds to an \verb$INT$ with an abort. In the
former cases, the dnode.sh script will terminate immediately and not
try to restart the dnode.

When initially started, the dnode has a `tiny' memory, just enough to
bootstrap itself. In this state, it uses minimal resources, including
cpu utilization.

\pagebreak[3]
So, to summarize:\nopagebreak

\textbf{Local Start}\\*[-1.5\baselineskip]
\begin{enumerate}
  \item open an xterm (terminal window)
  \item change to your directory: \verb$cd /mnt/Lab1/dm-3.0-arch/$.
  \item run \texttt{dnode.sh}: \verb$./dnode.sh 0$ (or n). 
\end{enumerate}

\textbf{SSH Start}\\*[-1.5\baselineskip]
\begin{enumerate}
  \item open an xterm
  \item make sure that you can access your local xserver, either with
  xhost +otherserver, or using ssh mechanisms (see section~\ref{sec:run}).
  \item \verb$ssh otherserver$.
  \item \verb$cd /mnt/Lab1/dm-3.0-arch/$
  \item \verb$./dnode.sh 0$ (or n).
\end{enumerate}


\section{Running a d-machine}
\label{sec:run}

Once the dvt and dnodes are started, the machine can be
integrated. The basic command for this is:

\verb$host port group memorysize _csu$
\begin{description}
  \item[host] is a string identifying the dnode host machine. It
  should be of the form: \verb$(klutz3.nonnerlabe)$. For G5's running
  OSX Panther, use \verb$(klutz3)$ with no domain appended. The name
  (localhost) generally does not work.
  \item[port] is the port number offset for that host (n), usually a
  small number like 0,1,10\dots, depending on the number of dnodes
  running on that host.
  \item[group] is a small negative number that puts several nodes into
  a `group' with that same number. This is a device for sending the
  same command to multiple nodes.
  \item[memorysize] is an array(5) of longs describing memory
  allocation as follows:
  \begin{description}
    \item[1: nopds] The size of the operand stack. Between 1000 and
    100000.
    \item[2: ndicts] The size of the dictionary stack. Between 100 and 10000.
    \item[3: nexecs] The size of the execution stack. Between 50 and 5000.
    \item[4: VM] The size of the VM in megabytes. Between 1 and 1000.
    \item[6: userdictsize]  The size of the initial userdict. Between
    200 and 5000.
  \end{description}
\end{description}

At the dnode, \verb$_csu$ calls \verb$vmresize$, which takes the final
parameter of \verb$_csu$ as its only parameter. It creates the memory,
and then \verb$_csu$ loads \texttt{startup\_dnode.d}. This initializes
the userdict for the dnode.

This file also initializes the X-connection to the dvt. An eye for the
dnode is popped up on the Xserver identified by \verb$/getmyname$ at
the dvt. This is initialized by default to the value of the
environmental variable \verb-$DISPLAY-,%$
or the local hostname if that is set to localhost:x, where x is the
display number.  Usually, this is all set automagically.  However,
this can be a security issue; this means that \verb$xhost +$ may have
to be called on the command line, or \verb$/getmyname$ may need to be
set to an ssh tunnel by hand, before calling \verb$_csu$.

At this point, several graphical changes should occur:
\begin{itemize}
  \item TheHorses should be populated with the new dnode. A new line
  should appear saying something like klutz3:0, for a dnode on the
  klutz3 on port 0.
  \item A new eye should appear labeled in the same way.
\end{itemize}

This can be repeated to connect several dnodes. In order to disconnect
from a dnode, call:

\verb$host _dc$, 

where host is the dnodes number in order of TheHorses interface. In
other words, the first dnode you connect to will be 1, the next 2,
\dots If you attempt to reconnect later, empty spaces will be filled,
and the number will be of its \emph{position}, not the order in which
it was created.  When a node is disconnected, its eye will continue on
the current screen.  It can, however, be destroyed by calling
\verb$null vmresize$ directly on the dnode before disconnection.

A dnode can be reconnected after a \verb$_csu$, \verb$_dc$ sequences
without reinitializing the memory by calling:

\verb$hostname port group _c$,

where hostname, port and group are the same as in \verb$_csu$.  This
can be particularly useful if the dvt dies. Operations on the dnodes
are not interrupted. They continue, and any output goes to their
standard error stream (probably the xterm on which they were started).
By starting a new dvt, and using \verb$_c$, the output is redirected
to the new dvt without interrupting pending operations. However, this
will not move the node's eye from its current screen.


\subsection{Using TheHorses}
\label{sec:horses}

To talk with the a dnode, TheHorses is the primary controlling
interface. By left-clicking on a dnode or the dvt, it is selected as
the primary keyboard owner. By clicking with button three (or
\verb$Ctrl$-button 1), the node and other nodes in its group are
selected as the keyboard owner.  All selected nodes appear in bold
blue text. An alternate method for selected a node is via:

\verb$target _t$,

where target is a node\#, or group\# (as in \verb$_dc$,
section~\ref{sec:dnode}).

While selected as the keyboard owner, all commands entered through
emacs are directed to those nodes. To send a command to the dvt, it
must be reselected as the keyboard owner. The command is sent to the
selected nodes when return is hit; if a command is entered, and then
the node selected is changed, and finally return is hit from the
keyboard, it will be sent to the newly selected node.

While a node is processing the sent command, the node is highlighted
in TheHorses. If it is selected as the current keyboard owner,
commands will be rejected when return is hit unless prepended by an
exclamation mark (\verb$!$). See the other manual for control keys
associated with this. 

When the operation is finished, the node will return to its
unhighlighted state. This lock mechanism is intended to avoid
accidentally interrupting a long-running process. If an error occurs
at the dnode, \verb$!stop$ sent to the node will usually end the
process, and set the node as ready. To force a node to appear to be
ready, either click the node with button 2, or shift-button1, or call
\verb$target setready$ in the dictionary \verb$dvt$.  This will clear
the busy flag, but will not actually send any command to the node. It
simply allows one to send a command without prepending a \verb$!$.

These operations are summarized in section~\ref{sec:shorthorses}.

TheHorses belongs to the dvt. Its operations occur within the dvt, and
do not `speak' directly to the dnodes. They only change the state of
the dvt, and thereby control the directing of keyboard input to the
proper node, subsequently.


\subsection{Tiny Window Manager}
\label{sec:twm}

All windows are managed by twm. The manager is fully documented by the
man pages (accessed by typing in emacs \verb$Alt-x man$ and then
\verb$twm$). However, a brief comment may be useful.

The d-machine windows, excluding emacs, will appear in the d\_machine
Icon Manager, which is initially on the upper right hand side of the
screen. Each row has the name of the associated window, and clicking
on it either minimizes or restores that window. Windows can be dragged
by the title bar. On the right side of the window is a resize symbol.
By dragging it past the edge of the window, one can resize windows;
however, most d-machine windows are not resizable.

The first symbol on the left-hand side of the title bar will
minimize/restore the window upon clicking. The next-symbol from the
left will pop-up a bar with the options: Iconify (minimize/restore),
Delete (request that the owning process kill the window), and Kill
(immediately remove the window).  Using kill will generally also kill
the associated process (kill your dnode).

Windows pop up randomly on the screen. You will have to place them in
some useful/rational order by dragging the title bars. 

\subsection{TheEye}
\label{sec:eye}

Now that the dnodes are connected and can be operated on via the
keyboard, TheEye can be analyzed. One exists for each dnode, in
addition to the dvt. It is important to note that when an operation is
done on an eye, a piece of d-code is placed on top of the execution
stack, interrupting current operation. This has two implications:
\begin{enumerate}
  \item While a d operation is within an operator (i.e., within a
  piece of c), TheEye will not respond until after its completion.
  \item At the end of an operation, TheEye's operation will begin with
  the current stack, dictionaries, etc\dots, unaltered below the level
  of the current operation. Generally this is safe, as TheEye
  operations encapsulate themselves, leaving the state of the node/dvt
  unchanged. However, don't push it too far; clicking quickly during
  important/time-consuming operations is a recipe for disaster.
\end{enumerate}

The eye is composed of:
\begin{description}
  \item[Left Pane:] stores complex objects that have been selected.
  Left-clicking puts identifying characteristics in the top bar.
  \verb$Alt$-click, or middle-click selects the object for macro
  operations (it turns red). \verb$Ctrl$-click, or right-click, opens
  the composite object in the right pane. \verb$Shift$-click removes
  the object from the left pane. Userdict and the filesystem root
  (\verb$/$) appear automatically in the left pane, and cannot be
  removed.
  \item[Right Pane:] displays all the members of the last opened
  composite object. Left-clicking puts identifying characteristics in
  the top bar. \verb$Ctrl$-click or right-click will place a composite
  object in the left-pane, and open it in the right pane.
  \verb$Alt$-click or middle-click will toggle the select state for
  the object for macro operations (it will turn bold blue or return to
  black). Multiple objects can concurrently be selected in the right
  pane.
  \item[Bottom Bar:] controls scrolling for the left and right panes;
  it is repeated twice, each one controlling the pane above it. The
  single-headed arrows scroll the window one line in the obvious
  direction. The double headed arrows scroll by a page. The double
  headed arrow with a bar moves to the beginning or end of the pane.
  \verb$Shift$-click or right-click will move proportionally, in terms
  of your position on the bar. In other words, right-clicking with the
  mouse in the middle of the bar will scroll to the middle of the
  pane, and right-clicking two-thirds to the right will scroll to
  two-thirds of the way through the pane.
  \item[Top Bar:] displays identifying information of a
  d-object/file which has been left-clicked. 
  \begin{description}
    \item[file:] size, modification time, and permission.
    \item[directory:] canonical path.
    \item[dictionary:] total elements, and current number of elements.
    \item[list:] elements.
    \item[array:] type and size. For byte arrays, the initial string is
    also shown.
    \item[boolean,number:] type and value.
  \end{description}
\end{description}

The shortcuts are summarized in section~\ref{sec:shorteye}.

Remember, each eye belongs to a separate node, which may be running
on a different machine. The dvt is not directly linked to the eyes of
the dnodes (only its own).  Your X-server is displaying it on your
screen, but the code that creates the window is running inside the
process of the associated dnode.  So, to access the state of an eye,
you must run code in the associated dnode.  This is important for the
interaction of the several windows, particularly the macro interface
(discussed in section~\ref{sec:macro}).

\subsection{Keyboard/Emacs}
\label{sec:keyboard}

Of course, the primary interface to the system is via the command
line, embedded within emacs \citep{dvt-emacs}. Once the connections
have been made, TheHorse, TheEye and the command line interact to
manage the operation of the entire d-machine.

A command entered if font-lock is on is initially colored by d-code
conventions. After \verb$return$ is hit, this is converted by emacs to
blue. Output from a node is by default in black text. In particular,
output from the dvt is always black. Errors from any node are in red.
This is encoded by the node using ansi color codes, and requires for
conversion into actual colors that
\verb$ansi-color-for-comint-term-on$ be called in emacs for the dvt
buffer. That command is called by default by the dvt-mode code.

A line entered is understood in terms of a preceding tag by the dvt,
which then feeds the phrase up to the newline, in the proper context,
to the current target (which may be itself):
\begin{description}
  \item[untagged] the phrase is sent to the keyboard owner, to be
  interpreted as d-code.
  \item[\texttt{!}] the phrase is decode to be sent to the keyboard
  owner, regardless of the current busy state.
  \item[\texttt{\$}] the phrase is a shell command to be executed by
  the keyboard owner.
  \item[\texttt{\#}] the phrase is interpreted as d-code by the dvt,
  regardless of owner. The string produced is then sent to the
  keyboard owner, to be executed as a shell command.
  \item[\texttt{@}] same as \verb$#$, but it is interpreted by the
  keyboard owner rather than the dvt.
\end{description}

For \verb$#$ and \verb$@$, the command line should be of the following form:
\begin{description}
  \item[\texttt{(string) fax}] appends a shell string literal.
  \item[\texttt{faxLpage}] appends the object selected in the left
  pane of the eye.
  \item[\texttt{faxRpage}] appends all selected objects in the right
  pane of the eye.
\end{description}
A string-buffer and index are placed on the stack before the operation
is called. Afterwards, \verb$0 exch getinterval$ is called and the
resulting string is executed as a shell command. For \verb$#$, this is
done in the context of the dvt's eye, and for \verb$@$ this is in term
of the keyboard owners eye.

In the dvt the shell command is spun off as a background process,
allowing the continued operation of the dvt. However, in a dnode it is
a foreground process, halting operation of the dnode until it has
terminated. Generally, operating system commands are most usefully
done directly from the dvt. The dvt and the dnodes are generally setup
to have parallel file structures; in other words, the
\texttt{/mnt/Lab*} are the same directories, either through nsf
network mounting or by having the dnode and the dvt on the same local
machine.

For normal d-code, the following operators are useful:
\begin{description}
  \item[\texttt{getLpage}] returns the object selected in the left
  pane of the eye.
  \item[\texttt{getRpage}] returns a list of objects selected in the
  right pane of the eye.
\end{description}
These commands of course occur in the context of the current keyboard
owner.

Further color-coding is available in the module
\texttt{/mnt/Lab1/dm-3.0-arch/color.d}. This modules creates two
operators in the userdict, \verb$color_fax$ and \verb$color_text$,
useful for adding color control codes to strings. They work both in
the dvt and in dnodes. Usage instructions are embedded in the module's
commentary.


\subsubsection{Color coding of d-node output}
\label{sec:color}

To have alternate colors for the output of each node, a command line
interface has been added to the dvt. The command \verb$color_node$ in
userdict takes a color list and a node\# (in terms of TheHorses), and
sets the color output of that node. The coloring is actually done in
the dnode.

{For example:\samepage
\begin{verbatim}
[/cyan /italic] 1 color_node
\end{verbatim}
  will set output all output from a \verb$/toconsole$ command on node 1
  to cyan and italic.}

{To remove coloring:\samepage
\begin{verbatim}
[/reset] 1 color_node
\end{verbatim}}
  
An alternate formulation for multiple nodes is: \\
\verb$[[/color_name...]...] color_nodes$.  This will call
\verb$color_node$ on each node from 1\dots n, where n is the length
of the list, with the respective color name list. Zero member color
lists will cause the command to skip the associated node (not call
\verb$color_node$ on that dnode).

The full list of color names are: 
\begin{description}
  \item[Foreground:] \verb$/black$, \verb$/red$, \verb$/green$,
  \verb$/yellow$, \verb$/blue$, \verb$/magenta$, \verb$/cyan$,
  \verb$/white$.
  \item[Background:] \verb$/on_black$, \verb$/on_red$,
  \verb$/on_green$, \verb$/on_yellow$, \verb$/on_blue$, \\
  \verb$/on_magenta$, \verb$/on_cyan$, \verb$/on_white$.
  \item[Styles:] \verb$/bold$, \verb$/faint$, \verb$/italic$,
  \verb$/underlined$, \verb$/slow_blink$, \verb$/rapid_blink$,
  \verb$/negative$. (Some of these are implemented fairly poorly by
  emacs).
\end{description}

Additionally, all of these operators will add a code the TheHorse,
following the hostname and port number. The letter B followed by a
single letter indicates the background color (\textbf{b}lack,
\textbf{r}ed, \textbf{g}reen, \textbf{y}ellow, b\textbf{l}ue,
\textbf{m}agenta, \textbf{c}yan, \textbf{w}hite). The letters will be
in a color similar to that produced on emacs. The letter F followed by
the same letter indicates the foreground color. Finally, the letter S
followed by a code indicates the style of the output (\textbf{i}talic,
\textbf{b}old, \textbf{f}aint, \textbf{\_} underlined, \textbf{s}low
blink, \textbf{r}apid blink, \textbf{n}egative), and will be styled in
a way suggestive of the output.

\subsection{DVT Macro}
\label{sec:macro}

An essential integrative window is the DVT macro bar. This bar runs in
the dvt proper, and sends to the emacs screen a string that can be
entered by the user to do a standard operation. Generally, this means
that you click on a command, choose something in the eye and then hit
\verb$up-cursor return$ or the \verb$f1$ key to send that string back
to the dvt \citep{dvt-emacs}.

The proper keyboard owner and matching eye must be used for many of
these command. For example, to load a module one would click on
\emph{Load} on the macro bar. The command:

\hspace{-0.6em}\verb$getRpage dup 0 get exch 1 get { exch dup 3 -1 roll fromfiles } forall pop$

will appear on the emacs command line. Select all modules to load in
the right pane of the dnode you want, and select that dnode as the
keyboard owner. Then hit \verb$f1$ or \verb$up-cursor return$ to enter
that command and thereby send it to the dnode. While the operation is
occurring you will see that dnode highlighted in TheHorses.

The operations functional under osx are:
\begin{description}
  \item[ChangeDir] changes the current working directory. By default,
  the node runs in \texttt{/mnt/Lab1/dm-3.0-arch}, and all operations
  are relative to that. This command uses the currently selected
  directory in the left pane of the current keyboard owners.
  \item[MakeDir] creates a directory. The selected directory in the
  left pane of the dvt eye is recreated on the current keyboard
  owners' host machines.
  \item[Remove] removes all files or directories selected in the right
  pane of the dvt from the current keyboard owners' hosts.
  \item[Copy] copies all files or directories selected in the right
  pane of the dvt to the directory selected on the left pane of the
  dvt, on the current keyboard owners' hosts.
  \item[Save] saves objects to the file chosen in the right pane,
  using the object generator selected in the left pane.  A literal
  directory path must be prepended by hand, such as
  \verb$(/mnt/Lab3/ic1/)$. This all occurs in the context of the
  current keyboard owners. (I haven't actually seen this one used.)
  \item[Load] loads modules from the files selected on the right
  pane. This also occurs in the context of current keyboard owners.
\end{description}

Additionally, under linux the following operators are available:
\begin{description}
  \item[CD:]\ \\*[-1.5\baselineskip]
  \begin{description}
    \item[Mount] mounts the cd drive to \texttt{/mnt/cdrom}.
    \item[Eject] ejects the cd drive.
  \end{description}
  \item[Stick:]\ \\*[-1.5\baselineskip]
  \begin{description}
    \item[Mount] mounts the the stick to \texttt{/mnt/usbmem}.
    \item[Unmount] unmounts the stick (which actually syncs the
    buffer).
  \end{description}
  \item[Printer Ops] are divided into two groups, \emph{PrintFrom} and
  \emph{PrintTo}. The operations in the first group print out the
  beginning of a print command, without a following newline. The
  second set completes the command, and adds a newline. So
  conceptually, the first group identifies the source type, and the
  second the destination. They print out the files currently selected
  in the right pane of the dvt, and generally are best used with the
  dvt as the keyboard owner. The \emph{PrintFrom} options are:
  \begin{description}
    \item[Talk] Directly connect to the print device. An xterm is
    created, which allows one to enter postscript commands to be
    executed by the printer or ghostscript.
    \item[Ascii] Send ascii files transformed by the shell command
    \texttt{a2ps} into postscript to the print device. That command
    nicely formats some known file types, like c code, and prints out
    in a two-page per physical page format.
    \item[PS] Send standard postscript directly to the print device.
    \item[xPS] Prepends the reporter-created postscript files with
    standard include files
    (\texttt{/mnt/Lab1/ps/\{form\_2.ps,text.ps,graf\_1.ps,struct.ps\}})
    before sending them to the print device.
  \end{description}
  The \emph{PrintTo} options are:
  \begin{description}
    \item[gs] Print to the ghostscript engine. This will pop up an
    xterm and a ghostscript window which will render the output.
    Hitting return will display page by page of the output.
    \item[LW] Print to \texttt{LaserWriter Pro 630}. The printer has
    to be manually reset (hit the power) often.
    \item[xpdf] Convert postscript to pdf (\texttt{file.ps} to
    \texttt{file.pdf}), and display the pdf with xpdf.
  \end{description}
  Finally, \emph{LWstatus} prints to the dvt terminal the current
  status of the laser writer (to check whether you need to hit the
  power button on it).
\end{description}

Additionally, if the module \texttt{color.d} is loaded into the dvt,
the macro commands will not be in the dvt's standard black, but will
appear in bold blue.

\section{Operator changes}

Some operators behave differently than in earlier implementations, or
have been added. These changes focus on tcp connectivity and the
multithreaded behavior of the d-machine.

\subsection{Operators \texttt{save} and \texttt{restore}}
\label{sec:save}

In earlier versions, \verb$restore$'ing an uncapped \verb$save$ would
result in the operand and dictionary stacks returning to their state
at the time of the original save. In the current implementation, this
often lead to segment violations. Therefore, the behavior of the
operator has been changed.

Now, the \verb$restore$ operator on an uncapped \verb$save$ acts in a
manner similar to a capped \verb$save$. In other words, the memory in
the box is deallocated without reverting the dictionary and operand
stacks. However, the \verb$restore$ on a capped \verb$save$ will fail
if either of those stacks contain reference into that region of memory
(such as an array on the stack which lives in the box). On the other
hand, the uncapped save will simply remove the offending frames from
the operand, dictionary and executable stacks.

\subsection{Operators \texttt{send}, \texttt{console}, and \texttt{connect}}
\label{sec:send}

A \verb$send$ operator has been added, which sends operators from the
dvt to the dnode and back. Its parameters are as follows:
\begin{verbatim}
socket [root (string)] |send| --
socket (string)        |send| --
\end{verbatim}
where socket is a null object encapsulating a socket number as
produced by \verb$connect$, root is a list (including procedure),
array or dictionary, and string is a piece of d code.

The (string) is executed on the foreign node. In the first form, the
root object is placed on the operand stack before executing the
(string).

Normally, the socket is retrieved on the dvt form \verb$nodelist$ in
the \verb$dvt$ dictionary (see the description of \verb$nodelist$ in
\texttt{startup\_dvt.d}). This value, however, is originally created
by the \verb$connect$ operator:
\begin{verbatim}
(servername) port# |connect| socket
\end{verbatim}

To simplify coding of dnode code, the \verb$console$ operator has been
added, which returns the socket of the dvt managing the dnode. This is
primarily for use in send, where most operation have a return send to
the dvt embedded in the command to be executed on the dnode.

For example, the following:
\begin{verbatim}
somenode (2 2 add _ console \(1 1 add _ \) send) send
\end{verbatim}
will add 2 and 2 on somenode, then send a command back to the
dvt to add 1 and 1.

\subsection{Tilde'd operators}
\label{sec:tilde}

Tilde'd operators have been added to simplify the creation of
procedures on the fly to send to dnodes or return to the dvt. A
name that has been tilded on the execution stack is neither placed on
the operand stack as a passive name, or executed on the execution
stack. Instead it is placed as an \emph{active} name on the
\emph{operand} stack. Likewise, a tilded mark creates an active list
(a procedure), of the elements between the mark and the close mark.

For example:\samepage
\begin{verbatim}
/somedict 2 dict dup begin 
  /a {(a1\n) toconsole} def 
  /b {(b1\n) toconsole} def 
end def
/a {(a2\n) toconsole} def 
/b {(b2\n) toconsole} def 
/test ~[ somedict ~begin a ~b ~end] def
/somedict 2 dict dup begin 
 /a {(a3\n) toconsole} def 
 /b {(b3\n) toconsole} def 
end
test
\end{verbatim}
will output \verb$a2$ during the definition of \verb$test$, and
\verb$b1$ when \verb$test$ is called. The operators \verb$a$ and
\verb$somedict$ are resolved and executed during the definition of
\verb$test$, but \verb$b$ is resolved and executed during the
execution of \verb$test$.  Additionally, the resolution of \verb$b$ is
in context of the \verb$somedict$ resolved during the definition,
instead of the \verb$somedict$ that would normally be resolved during
execution.

This can be most useful in locutions involving \verb$send$ (subsection
\ref{sec:send}), as follows:\samepage
\begin{verbatim}
/n 1 def
somenode [~[~nodedict ~begin n ~n ~add ~_ ~pop] (exec)] send
\end{verbatim}

This will add the current value of \verb$n$ in the dvt (1), with the value
of n in nodedict on the dnode (\verb$somenode$), and display it.

\section{Shortcuts}
\label{sec:short}

Most graphical commands can be entered in one of three ways: by a
piece of dvt code, by a one-button mouse combined with control, alt,
or shift, or by a three-button mouse click.

\textbf{NB:} On G5 Panther machines, all references to Alt key must be
replaced with the Apple key.

\subsection{TheHorses shortcuts}
\label{sec:shorthorses}

\begin{tabular}{|l|c|c|c|}
  \hline
  Operation & DVT command & 1-button & 3-button \\
  \hline
  Select Node & \verb$target _t$ & Click & 1-click \\
  Clear Busy & \verb$target setready$ & Shift & 2-click \\
  Select Group & \verb$-target _t$ & Ctrl & 3-click \\
  \hline
\end{tabular}

\subsection{TheEye shortcuts}
\label{sec:shorteye}
\textbf{Left Pane:}\nopagebreak

\begin{tabular}{|l|c|c|}
  \hline
  Operation & 1-button & 3-button \\
  \hline
  Get Info & Click & 1-click \\
  Select & Alt & 2-click \\
  Open  & Ctrl & 3-click \\
  Remove  & Shift & Shift-1-click \\
  \hline
\end{tabular}
\bigskip


\textbf{Right Pane:}\nopagebreak

\begin{tabular}{|l|c|c|}
  \hline
  Operation & 1-button & 3-button \\
  \hline
  Get Info & Click & 1-click \\
  Select & Alt & 2-click \\
  Open & Ctrl & 3-click \\
  \hline
\end{tabular}
\bigskip

\textbf{Bottom Bar}:\nopagebreak

\begin{tabular}{|l|c|c|c|}
  \hline
  Operation & Location & 1-button & 3-button \\
  \hline
  Scroll 1 line & $<$ or $>$ & click & 1-click \\
  Scroll 1 page & $<\!\!<$ or $>\!\!>$ & click & 1-click \\
  Scroll to end & $|\!\!\!<\!\!<$ or $>\!\!>\!\!\!|$ & click & 1-click \\
  Scroll proportionally & proportional position & Shift-click &
  3-click \\
  \hline
\end{tabular}
\bigskip

\subsection{Emacs shortcuts}
\label{sec:shortemacs}

These are described in \citet{dvt-emacs}.


\subsection{DVT macros}
\label{sec:shortmacro}

The macro bar functions by left-click or single-click. The text
produced on the terminal then must be sent to the keyboard owner with
an \verb$up-cursor return$ or by \verb$f1$. The commands available are
explained in section~\ref{sec:macro}.

\subsection{X-windows}

Under aqua on osx and full-screen X, \verb$Ctrl-Apple-A$ will switch
between the X and aqua desktops.

\bibliography{dvt}

\end{document}
