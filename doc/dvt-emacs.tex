\documentclass[12pt]{article}

\usepackage{pdf}
\usepackage{hyperref}

\usepackage{natbib}
\bibliographystyle{plainnat}

\usepackage{redefs}
\renewcommand{\bibsection}{%
  \section*{\refname}%
  \addcontentsline{toc}{section}{\refname}%
}

\usepackage{verbdef}
\verbdef\id$Id$
\usepackage{config}

\title{
  Emacs \& D-machine: d source files and dvt\\
  \titleversion
}

\author{Alex Peyser}

\begin{document}
\pagenumbering{roman}
\maketitle
\vspace{\stretch{1}}\tableofcontents\vspace{\stretch{1}}
\newpage
\setlength{\parindent}{0pt}
\setlength{\parskip}{6pt}
\pagenumbering{arabic}
\section{Introduction \& Installation}

The emacs interface to the dvt is composed of a set of lisp files:
\begin{verbatim}
dm-3.0/
  site-start.el
  ap/
     d-comint-mode.el
     d-mode.el
\end{verbatim}
\texttt{site-start.el} should be linked from
\texttt{/usr/share/emacs/site-lisp/site-start.el} and \texttt{ap}
should be linked from \texttt{/usr/share/emacs/site-lisp/ap} in order
to be properly loaded by default. The file \texttt{site-start.el}
defines the automatic loading of the other files and sets some
sensible defaults.
  
Most emacs commands can either be called from a file in the form
\emph{command param1 param2 \dots}, such as in your
{\tt\verb$~/.$emacs} file for defaults executed on load, or by typing
\verb$Alt-x$ \emph{command} (you will then be prompted for
parameters).

The functionality of the d-machine version 3.0, as opposed to the
emacs specific interface, is specified in more detail in ``D-machine
3.0 Extensions, and interface through DVT,'' \citep{dvt-3.0}.

\section{D source files}
  
The lisp file \texttt{d-mode.el} defines the major mode for editing d
code, which is by default defined to be any file of the form
\verb$*.[dD]$.  Files will also be recognized as d files if they begin
with the line:
\begin{verbatim}
| -*- mode: D; -*-
\end{verbatim}
Also, d-mode can be activated on an already loaded file with the interactive
command \verb$Alt-x d-mode$.

When a d file is loaded, the file is scanned for carriage returns and
if any are found, the user will be queried ``Eliminate carriage
returns?'' Answering `y' will remove the carriage returns and replace
them with newlines if necessary. The user will then be asked whether
the file should be immediately saved in its new unix standard format.
On the other hand, if `n' is the response to the first query, the file
will be edited as is and newlines will be inserted according to the
convention of the file, that is, mac or dos formats.

\subsection{Fontification}
  
D mode can fontify text according to emacs standards for programming
languages. This is activated in the normal way for emacs, by either
calling \verb$global-font-lock$ which takes \verb$t$(activates
font-lock) or \verb$nil$ as a parameter or toggles when called
interactively.  To change the font-lock-mode for a specific file, call
\verb$font-lock-mode$ interactively which toggles the font-lock for
the current buffer, or define a function for a mode in general as
follows in your \texttt{.emacs}:
\begin{verbatim}
(defun my-d-mode-hook ()
  (font-lock-mode t))
(add-hook 'd-mode-hook my-d-mode-hook)
\end{verbatim}
Replacing \verb$t$ with \verb$nil$ will turn off font-lock-mode for
d-mode.

The categories for d-mode font-lock are as follows:
\begin{description}
  \item [font-lock-keyword-face] built in operators, such as
  \verb$list$, \verb$array$\ldots
  \item [font-lock-variable-name-face] any other legal d name which is
  not preceded by \verb$/$ or \verb$~$
  \item [d-mode-oref-face] \verb$~$\emph{name}
  \item [font-lock-reference-face] \verb$/$\emph{name}
  \item [font-lock-builtin-face] a constant such as a number, or
  \verb$true$/\verb$false$
  \item [font-lock-type-face] any of \verb$~[]{}$
  \item [font-lock-comment-face] any comment from \verb$|$ to newline
  \item [font-lock-string-face] anything between \verb$($ and
  \verb$)$, or \verb$<b/w/l/s >$
\end{description}
  
These can be customized by using the standard emacs font-lock
customizer or by setting lines of the following kind in your
\texttt{.emacs}:
\begin{verbatim}
(set-face-foreground 'font-lock-keyword-face "red")
(make-face-bold 'font-lock-keyword-face)
\end{verbatim}
There is some interference between the comment face and the string
face, so escaping unmatched parens in comments is suggested as
follows:
\begin{verbatim}
| Here is an escaped paren \( and here is a pair ()
| but the following paren may mess up your font lock (
\end{verbatim} %\)
\subsection{Minor modes}
  
D-mode has 4 associated minor modes, which are off by default. Each
has letter associated with it (such as p) which appears on the mode
line following \verb$D Machine-$ when activated.  That letter is also
an element of the key sequence which toggles the mode.  For example,
\verb$C-c C-p$ toggles d-mode-magic-parens, and a \verb$p$ appears on
the mode line when it is active.  Modes are:
\begin{description} 
  \item [d-mode-magic-parens(p):] typing a mark (such as \verb${}[]$)
  will re-indent the line to match the preceding line for openers or
  the matching opener's line for closers.
      
  \item [d-mode-magic-newline(n):] hitting \verb$return$ will start
  the cursor at the proper indent position for the next line. Lines
  will indent the same as the previous line unless it has an unmatched
  opening mark, in which case it will indent \verb$tab-width$
  characters further.
  
  \item [d-mode-magic-comments(c):] hitting `\verb$|$' will place the
  comment mark in the same column as the comment mark for the last
  preceding comment.
      
  \item [d-mode-magic-delete(d):] hitting \verb$backspace$ (the big
  delete key on mac keyboards) will delete at least one character, but
  if the current cursor position is surrounded multiple spaces
  including newlines \& tabs, all preceding spaces will be deleted
  unless that would join two seperate words, in which case all but one
  will be deleted. For example: (Cursor positon will be marked by
  \verb$|$)

\begin{verbatim*}
one two three     |four five
one two three |four five

one two three  |  four five
one two three|  four five   
\end{verbatim*}
\end{description}

\subsection{Example \tt.emacs}
  
An example hook for d-mode in {\tt\verb$~/.$emacs} which will turn
everything on and set an indent increment of 4 characters:
\begin{verbatim}
(defun my-d-mode-hook ()
  (setq d-mode-magic-parens t)
  (setq d-mode-magic-newline t)
  (setq d-mode-magic-comment t)
  (setq d-mode-magic-delete t)
  (setq tab-width 2)
  (font-lock-mode t))

(add-hook 'd-mode-hook 'my-d-mode-hook)
\end{verbatim}
\subsection{Tabs}
  
Tabs will indent the current line, as defined previously.  To insert
hard spaces regardless of the indent rules, press
\verb$Shift-LeftTab$ or \verb$Alt-Tab$.  Indenting works as per emacs
standard, controlled by the commands \verb$tab-always-indent$ and
other parameters in the `indent' lisp package.

\section{DVT interface}
  
The other emacs mode associated with the d-machine is d-comint-mode.
This mode is started by calling the command \verb$Alt-x dvt$, which
will start a dvt shell running a dvt from
\texttt{/mnt/Lab1/dm-3.0-arch/dvt}, where arch is g4, g5 or osx
depending on the current machine type. The location and name of the
dvt to be started can be changed by setting
\verb$explicit-dvt-file-name$ in your \texttt{.emacs} file, or the
environmental variables \verb$EDVT$ or \verb$DVT$. The order of
precedence is \verb$explicit-dvt-file-name$, \verb$EDVT$, \verb$DVT$,
\texttt{/mnt/Lab1/dm-3.0/dvt}.

Do it in your \texttt{.emacs} as follows:
\begin{verbatim}
(setq explicit-dvt-file-name "/home/juan/dvt-dir/dvt")
\end{verbatim}
By default the buffer will be named \texttt{*dvt*}, and if it is
already running when dvt is called it will switch your current buffer
to that buffer, rather than starting a new one. In order to start a
dvt buffer with a different name, type \verb$C-u Alt-x dvt$ instead
of \verb$Alt-x dvt$, and you will be prompted for the buffer name.

\subsection{Fontification}

DVT mode shares d-mode fontification on the command line typed by the
user, but it will not fontify text send by the dvt process to emacs.
Also, dvt mode turns on ansi-color-for-comint-mode, converting
standard ansi color sequences sent by the dvt into actual color.
Output from the dvt is by default black (has the face
d-comint-mode-highlight-output), and the input is blue
(comint-hightlight-input), though of course if font-lock-mode is on
the command line is initially fontified by d-mode.  Errors are output
as red.

\subsection{Minor modes}
  
One minor mode for the dvt is d-comint-mode-scream.  It is toggled by
\verb$control-!$ or \verb$control-1$, and places a \verb$!$ in the
mode line. When it is activated, any lines typed by the user will have
a \verb$!$ prepended to it if it does not already have one.  This is
for sending commands to busy dnodes, particularly useful when
debugging a dnode.
  
Another minor mode for the dvt is d-comint-mode-redirect-mode.  It is
toggled by \verb$control->$, and places a \verb$>$ in the mode line.
All output from the dvt is redirected from the screen and is appended
to a file with the same name as the current buffer.

\subsection{Key bindings}

Key binding for dvt:
\begin{description}   
  \item [\large\tt f1] sends the preceding line to the currently
  selected dmachine (useful for grabbing macros printed by the dvt
  process to the buffer, and sending it back to the currently selected
  node as a command)
  \item [\large\tt f2] sends \verb$! continue$ to the currently
  selected dmachine and turns off scream mode if active
  \item [\large\tt f3] sends \verb$! stop$ to the currently selected
  dmachine and turns off scream mode if active
  \item [\large\tt f4] sends \verb$! abort$ to the currently selected
  dmachine, and turns off scream mode if active
  \item [\tt{\large C-!} {\rm or} {\large C-1}] toggles scream mode
  \item [\large\tt C-c C-c] clears everything in the buffer from the
  current cursor position to the beginning of the buffer
  \item [\large\tt C-c C-a] behaves like \verb$f1$, except the command
  is wrapped in \verb${} debug_abort$, for debugging dvt process d
  code.  See \texttt{startup\_dvt.d} for \verb$debug_abort$'s
  functionality.
  \item [\large\tt C->] toggles redirect-mode 
\end{description}

For detail regarding selecting a keyboard-owning node or nodes, see
\citet{dvt-3.0}.

\bibliography{dvt}

\end{document}

